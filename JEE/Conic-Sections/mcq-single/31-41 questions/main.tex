
\iffalse
  \title{Conic-section}
  \author{Malakala bala subrahmanya aravind}
  \section{mcq-single}
\fi

%   \begin{enumerate}
\item A hyperbola passes through point $\vec{P}\brak{\sqrt2,\sqrt2}$  and  has  foci  at $\brak{\pm2,0}$. Then  the  tangent  to  this  hyperbola at $\vec{P}$ also passes through the point :
      \hfill{(JEE M 2017)} 
	\begin{enumerate}
    		\item  $\brak{-\sqrt2,-\sqrt3}$
    		\item  $\brak{3\sqrt2,2\sqrt3}$
    		\item  $\brak{2\sqrt2,3\sqrt3}$
    		\item  $\brak{\sqrt3,\sqrt2}$
	\end{enumerate}
\item  The radius of a circle, having minimum area, which touches the curve $y=4-x^2$ and the lines, $y=\abs{x}$ is : 
   \hfill{(JEE M 2018)}
	\begin{enumerate}
     		\item $4\brak{\sqrt2+1}$
     		\item $2\brak{\sqrt2+1}$
     		\item $2\brak{\sqrt2-1}$
     		\item $4\brak{\sqrt2-1}$
	\end{enumerate}
\item Tangents are drawn to the hyperbola $4x^2-y^2=36$ at the points $\vec{P}$ and $\vec{Q}$. If  these tangents intersect  at the point $\vec{T}\brak{0,3}$ then the area (in sq.units) of $\Delta$ PTQ is:
     \hfill{(JEE M 2018)}
	\begin{enumerate}
     		\item $54\sqrt3$
     		\item $60\sqrt3$
     		\item $36\sqrt3$ 
     		\item $45\sqrt5$
	\end{enumerate}
\item Tangent and normal are drawn at $\vec{P}\brak{16,16}$ on the parabola $y^2=16x$,
which is intersect the axis of the parabola at $\vec{A}$ and $\vec{B}$, respectively. If $\vec{C}$ is the centre of the circle through the points $\vec{P}$, $\vec{A}$ and $\vec{B}$ and $\angle$ CPB=$\theta$, then the value of $\tan{\theta}$ is :
     \hfill{(JEE M 2018)}
	\begin{enumerate}
    		\item $2$
    		\item $3$
    		\item $\frac{4}{3}$
    		\item $\frac{1}{2}$
	\end{enumerate}
\item  Two sets $A$ and $B$ are as under:
$A=\cbrak{\brak{a,b}\in\mathbb{R}\times\mathbb{R}:\abs{a-5}<1\text{ and} \abs{b-5}<1}$
$B=\cbrak{\brak{a,b}\in\mathbb{R}\times\mathbb{R}:4(a-6)^2\text{+}9(b-5)^2\leq36}$
    \hfill{(JEE M 2018)}
	\begin{enumerate}
		\item $A\subset B$ 
		\item $A\cap B$
		\item neither $A\subset B$ nor $B\subset A$
		\item $B\subset A$
	\end{enumerate}
\item If the tangent at $\brak{1,7}$ to the curve $x^2=y-6$ touches the circle $x^2+y^2+16x+12y+c=0$ then the value of c is :
       \hfill{(JEEM 2018)}
	\begin{enumerate}
    		\item $185$
    		\item $85$
    		\item $95$
    		\item $195$
	\end{enumerate} 
\item Axis of a parabola lies along X-axis. If its vertex and focus are at a distance $2$ and $4$ respectively from origin, on the positive X-axis then which of the following points does not lie on it? 
     \hfill{(JEE M 2018)} 
	\begin{enumerate}
    		\item $\brak{5,2\sqrt6}$
    		\item $\brak{8,6}$
    		\item $\brak{6,4\sqrt2}$
    		\item $\brak{4,-4}$
	\end{enumerate}
\item Let $0<\theta<\pi/2$. If the eccentricty of the hyperbola $\frac{x^2}{\cos^2{\theta}} - \frac{y^2}{\sin^2{\theta}} = 1$ is greater than $2$, then the length of its latus rectum lies in the interval:
          \hfill{(JEE M 2019-9 Jan(M)}
	\begin{enumerate}
    		\item $\brak{5,\infty}$
    		\item $\lbrak{\frac{3}{2}},\rsbrak{3}$
    		\item $\lbrak{2},\rsbrak{3}$ 
    		\item $\lbrak{1},\rsbrak{\frac{3}{2}}$
	\end{enumerate} 
\item Equation of a common tangent to the circle $x^2+y^2-6x=0$ and the parabola $y^2=4x$, is:
     \hfill{( JEE M 2019-9 Jan(M))}
	\begin{enumerate}
    		\item $2\sqrt{3}y=12x+1$ 
    		\item $\sqrt{3}y=x+3$
    		\item $2\sqrt{3}y=-x-12$ 
    		\item $\sqrt{3}y=3x+1$
	\end{enumerate}   
\item If the line $y=mx+7\sqrt{3}$ is normal to the hyperbola $\frac{x^2}{24}$-$\frac{y^2}{18}$ then a value of m is: 
     \hfill{(JEEM 2019-9 April(M))}
	\begin{enumerate}
    		\item $\frac{\sqrt{5}}{2}$ 
    		\item $\frac{\sqrt{15}}{2}$
    		\item $\frac{2}{\sqrt5}$
    		\item $\frac{3}{\sqrt5}$
	\end{enumerate}
\item If one end of a focal chord of the parabola, $y^2=16x$ is at $\brak{1,4}$,then the length of this focal chord is :
     \hfill{( JEE M 2019-9 Jan(M))}
	\begin{enumerate}
    		\item $25$
    		\item $22$
    		\item $24$
    		\item $20$
	\end{enumerate}    

% \end{enumerate}
