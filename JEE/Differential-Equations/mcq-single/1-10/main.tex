\iffalse
\title{CHAPTER - 19\\Differential Equations}
\author{AI24BTECH11028 - Ronit Ranjan}
\section{mcq-single}
\fi

%\begin{enumerate}
\item A solution of the differential equation \hfill \brak{1999 - 2 Marks}
\begin{align}
\brak{\frac{dy}{dx}}^2 - x \frac{dy}{dx} + y = 0 
\end{align}
\begin{multicols}{2}
\begin{enumerate}
    \item $y = 2$
    \item $y = 2x -4$
    \item $y = 2x$
    \item $y = 2x^{2}-4$
\end{enumerate}
\end{multicols}

\item If $x^2 + y^2 = 1$, then \hfill \brak{2000S}
\begin{multicols}{2}
\begin{enumerate}
    \item  $yy^{\prime\prime}-2\brak{y^\prime}^{2}+1 = 0$
    \item  $yy^{\prime\prime}+\brak{y^\prime}^{2}+1 = 0$
    \item  $yy^{\prime\prime}-\brak{y^\prime}^{2}-1 = 0$
    \item  $yy^{\prime\prime}+2\brak{y'^\prime}^{2}+1 = 0$
\end{enumerate}
\end{multicols}

\item If $y\brak{t}$ is a solution of $\brak{1 + t} \frac{dy}{dt} - ty = 1$ and $y\brak{0} = -1$, then $y(1)$ is equal to\hfill \brak{2003S}
\begin{multicols}{2}
\begin{enumerate}
    \item $-\frac{1}{2}$
    \item $e - \frac{1}{2}$
    \item $e + \frac{1}{2}$
    \item $\frac{1}{2}$
\end{enumerate}
\end{multicols}

\item If $y = y\brak{x}$ and $\frac{2 + \sin x}{y + 1}\brak{\frac{dy}{dx}} = -\cos x$, $y\brak{0} = 1$, then y\brak{\frac{\pi}{2}} \hfill \brak{2004S}

\begin{multicols}{2}
\begin{enumerate}
    \item $\frac{1}{3}$
    \item $-\frac{1}{3}$
    \item $\frac{2}{3}$
    \item $1$
\end{enumerate}
\end{multicols}

\item If $y = y(x)$ and it follows the relation $x \cos y + y \cos x = \pi$ then $y^{\prime\prime}(0) =$ \hfill \brak{2005S}

\begin{multicols}{2}
\begin{enumerate}
    \item $1$
    \item $-1$
    \item $\pi - 1$
    \item $-\pi$
\end{enumerate}  
\end{multicols}

\item The solution of primitive integral equation $\brak{x^2 + y^2}dy = xy \, dx$ is $y = y(x)$. If $y(1) = 1$ and $x_0 = e$, then $x_0$ is equal to \hfill \brak{2005S}

\begin{multicols}{2}
\begin{enumerate}
    \item $\sqrt{2\brak{e^2 - 1}}$
    \item $\sqrt{3e}$ 
    \item $\sqrt{2\brak{e^2 + 1}}$
    \item $\sqrt{\frac{e^2 + 1}{2}}$
\end{enumerate}
\end{multicols}

\item For the primitive integral equation $y dx + y^2 dy = x \, dy$; $x \in \mathbb{R}, y > 0, y = y(x), y(1) = 1$, then $y(-3)$ is \hfill \brak{2005S}

\begin{multicols}{2}
\begin{enumerate}
    \item $3$
    \item $1$ 
    \item $2$
    \item $5$
\end{enumerate}
\end{multicols}
`
\item The differential equation $\frac{dy}{dx} = \frac{\sqrt{1-y^2}}{y}$ determines a family of circles with \hfill \brak{2005S}
\begin{enumerate}
    \item variable radii and a fixed centre at \brak{0,1}
    \item variable radii and a fixed centre at \brak{0, -1}
    \item fixed radius 1 and variable centres along the x-axis
    \item fixed radius 1 and variable centres along the y-axis
\end{enumerate}
\item The function $y = f(x)$ is the solution of the differential equation \hfill \brak{JEE Adv.  2014}
\begin{align}
\frac{dy}{dx} + \frac{xy}{x^2-1} = \frac{x^4 + 2x}{\sqrt{1-x^2}}
\end{align}
in $(-1, 1)$ satisfying $f(0) = 0$. Then
\begin{align}
\int_{-\frac{\sqrt{3}}{2}}^{\frac{\sqrt{3}}{2}} f(x) \, dx is
\end{align}

\begin{multicols}{2}
\begin{enumerate}
    \item $\frac{\pi}{3} - \frac{\sqrt{3}}{2}$
    \item $\frac{\pi}{6} - \frac{\sqrt{3}}{4}$
    \item $\frac{\pi}{3} - \frac{\sqrt{3}}{4}$ 
    \item $\frac{\pi}{6} - \frac{\sqrt{3}}{2}$
\end{enumerate}
\end{multicols}

\item If $y = y(x)$ satisfies the differential equation\hfill \brak{JEE Adv. 2018}
\begin{align}
8\sqrt{x} \brak{\sqrt{9+\sqrt{x}}} dy = \brak{\sqrt{4+\sqrt{9+\sqrt{x}}}}^{-1} dx, \, x>0
\end{align}

and y(0) = $\sqrt{7}$, \text{ then } y(256) = 

\begin{multicols}{2}
\begin{enumerate}
    \item $3$
    \item $16$ 
    \item $9$
    \item $80$
\end{enumerate}
\end{multicols}
%\end{enumerate}
