
\iffalse
\title{straight lines}
\section{mcq-single}
\author{AI24BTECH11018 - Sreya}
\fi
%\begin{enumerate}
\item The points $\brak{-a,-b}$, $\brak{0,0}$, $\brak{a,b}$ and $\brak{a^2,ab}$\hfill{$\brak{1979}$}
\begin{enumerate}
      \item Collinear
    \item  Vertices of a parallelogram
    \item Vertices of a rectangle
    \item None of these
\end{enumerate}
\item  The point $\brak{4,1}$ undergoes the following three transformations successively.
\hfill{$\brak{1980}$}
\begin{enumerate}
    \item Reflection about the line $y=x$.
    \item Translation through a distance $2$ units along the positive direction of x-axis.
    \item Rotation through an angle $\frac{\pi}{4}$ about the origin in the counter clockwise direction.
\end{enumerate}
Then the final position of the point is given by the coordinates.
\begin{enumerate} 
    \item $\brak{\frac{1}{\sqrt{2}},\frac{7}{\sqrt{2}}}$
    \item $\brak{-\sqrt{2},7\sqrt{2}}$
    \item $\brak{\sqrt{2},7\sqrt{2}}$
    \item $\brak{-\frac{1}{\sqrt{2}},\frac{7}{\sqrt{2}}}$
\end{enumerate}
\item The straigth lines $x+y=0$,$3x+y-4=0$,$x+3y-4=0$ form a triangle which is \hfill{$\brak{1983-1 Mark}$}
\begin{enumerate}   
     \item isosceles
     \item equilateral
     \item right angled
     \item none of these
\end{enumerate}
\item if P=$\brak{1,0}$, Q=$\brak{-1,0}$ and R=$\brak{2,0}$ are three given points, then the locus of point S satisfing the relation\\$SQ^2+SR^2=2SP^2$, is 
\hfill{$\brak{1988-2 Marks}$}
\begin{enumerate}
    \item a straigth parallel to x-axis
    \item a circle passing through the origin
    \item a circle with the center at the origin 
    \item a starigth line parallel to y-axis 
\end{enumerate}
\item Line $L$ has intercepts $a$ and $b$ on the coordinate axes.When the axes are rotated through a given angle,keeping the origin fixed,line $L$ has intercepts $p$ and $q$ then
\hfill{$\brak{1990-5 Marks}$}
\begin{enumerate}
    
    \item $a^2+b^2=p^2+q^2$
    \item $\frac{1}{a^2}+\frac{1}{b^2}=\frac{1}{p^2}+\frac{1}{q^2}$
    \item $a^2+p^2=b^2+q^2$
    \item $\frac{1}{a^2}+\frac{1}{p^2}=\frac{1}{b^2}+\frac{1}{q^2}$
    
\end{enumerate}
\item If the sum of distances of a point from two perpendicular lines in a plane is $1$,then its locus is
\hfill{$\brak{1992-2 Marks}$}
\begin{enumerate}
    
        \item square
        \item straigth line
        \item circle
        \item two intersecting lines
    
\end{enumerate}
\item The locus of a variable point whose distance from $\brak{-2,0}$ is $\frac{2}{3}$ times its distance from the line $x=-\frac{9}{2}$ is
\hfill{$\brak{1994}$}
\begin{enumerate}
    
       \item ellipse
       \item hyperbola
       \item parabola
       \item none of these
    
\end{enumerate}
\item The equations to a pair of opposite sides of parallogrqam are $x^2-5x+6=0$ and $y^2-6y+5=0$, the equations to its diagnaols are 
\hfill{$\brak{1994}$}
\begin{enumerate}
    
      \item $x+4y=13,y=4x-7$
    \item $4x+y=13,y=4x-7$
      \item $4x+y=13,4y=x-7$
        \item $y-4x=13,y+4x=7$
    
\end{enumerate}
\item The orthocenter of the triangle formed by the lines $xy=0$ and $x+y=1$ is 
\hfill{$\brak{1995S}$}
\begin{enumerate}
    
      \item $\brak{\frac{1}{2},\frac{1}{2}}$
      \item $\brak{\frac{1}{3},\frac{1}{3}}$
      \item $\brak{0,0}$
      \item $\brak{\frac{1}{4},\frac{1}{4}}$
    
\end{enumerate}
\item Let $PQR$ be a right angled triangle, right at P$\brak{2,1}$. If the equation of the line $QR$ is $2x+y=3$.then the equation representing the pair of lines $PQ$ and $PR$ is
\hfill{$\brak{1990-2 Marks}$}
\begin{enumerate}
    \item $3x^2-3y^2+8xy+20x+10y+25=0$
    \item $3x^2-3y^2+8xy-20x-10y+25=0$
    \item $3x^2-3y^2+8xy+10x+15y+20=0$
    \item $3x^2-3y^2-8xy-10x-15y-20=0$
\end{enumerate}
\item If $x_1$,$x_2$,$x_3$ as well as $y_1$,$y_2$,$y_3$, are in G.P with the same common ratio then the points ($x_1$,$y_1$),($x_2$,$y_2$) and ($x_3$,$y_3$).
\hfill{$\brak{1999-2 Marks}$}
\begin{enumerate}
    
        \item lie on a straight line
        \item lie on ellipse
        \item lie on circle
        \item are vertices of a triangle 
    
\end{enumerate}
\item Let PS be the median of the triangle with vertices P$\brak{2,2}$, Q$\brak{6,-1}$ and R$\brak{7,3}$.The equation of the line passing through $\brak{1,-1}$ and parallel to PS is 
\hfill{$\brak{2000S}$}
\begin{enumerate}

      \item $2x-9y-7=0$  
      \item $2x-9y-11=0$
      \item $2x+9y-11=0$
      \item $2x+9y+7=0$

\end{enumerate}
\item The incentre of the triangle with vertices $\brak{1,\sqrt{3}}$,$\brak{0,0}$ and $\brak{2,0}$ is 
\hfill{$\brak{2000S}$}
\begin{enumerate}
    
     \item $\brak{1,\frac{\sqrt{3}}{2}}$
     \item $\brak{\frac{2}{3},\frac{1}{\sqrt{3}}}$
     \item $\brak{\frac{2}{3},\frac{\sqrt{3}}{2}}$
     \item $\brak{1,\frac{1}{\sqrt{3}}}$
    
\end{enumerate}
\item the number of integer values of $m$,for which the x-coordinate of the point of intersection of the lines $3x+4y=9$ and $y=mx+1$ is also an integer, is 
\hfill{$\brak{2001S}$}
\begin{enumerate}
    
    \item $2$
    \item $0$
     \item $4$
     \item $1$
    
\end{enumerate}
\item Area of the parallelogram formed by the lines $y=mx$,$y=mx+1$,$y=nx$ and $y=nx+1$ equals
\hfill{$\brak{2001S}$}
\begin{enumerate}
    
     \item $\frac{\abs{m+n}}{\brak{m-n}^2}$
     \item $\frac{2}{\abs{m+n}}$
     \item $\frac{1}{\abs{m+n}}$
     \item $\frac{1}{\abs{m-n}}$
    
\end{enumerate}
\item Let $0$\textless $\alpha $\textless $\frac{\pi}{2}$ be fixed angle.if\\
P=$\brak{\cos\theta,\sin\theta}$ and Q=$\brak{\cos\brak{\alpha-\theta},\sin\brak{\alpha-\theta}}$,then $Q$ is obtained from $P$ by
\hfill{$\brak{2002S}$}
\begin{enumerate}
    \item clockwise rotation around origin through an angle $\alpha$
    \item anticlockwise rotation around the orgin through an angle $\alpha$
    \item reflection in the line through origin with the slope $\tan\alpha$
    \item reflection in the line through origin with slope $\tan\brak{\frac{\alpha}{2}}$
\end{enumerate}
\item Let P=$\brak{-1,0}$,Q=$\brak{0,0}$ and R=$\brak{3,\sqrt{3}}$ be three points. Then the equation of the bisector of the angle PQR is 
\hfill{$\brak{2002S}$}
\begin{enumerate}
    
        \item $\frac{\sqrt{3}}{2}x+y=0$
        \item $x+\sqrt{3}y=0$
        \item $\sqrt{3}+y=0$
        \item $x+\frac{\sqrt{3}}{2}y=0$
    
\end{enumerate}
\item A straigth line through the origin $O$ meets the parallel lines $4x+2y=9$ and $2x+y+6=0$ at points Pand $Q$ respectively. Then the point $O$ divides the segment $PQ$ in the ratio 
\hfill{$\brak{2002S}$}
\begin{enumerate}

     \item $\ratio{1}{2}$
     \item $\ratio{3}{4}$
     \item $\ratio{2}{1}$
     \item $\ratio{4}{3}$

\end{enumerate}
%\end{enumerate}


