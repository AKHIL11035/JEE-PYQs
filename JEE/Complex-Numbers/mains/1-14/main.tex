\iffalse
\documentclass[journal]{IEEEtran}
\usepackage[a5paper, margin=10mm]{geometry}
%\usepackage{lmodern} % Ensure lmodern is loaded for pdflatex
\usepackage{tfrupee} % Include tfrupee package

\let\negmedspace\undefined
\let\negthickspace\undefined
\usepackage{gvv-book}
\usepackage{gvv}
\usepackage{cite}
\usepackage{amsmath,amssymb,amsfonts,amsthm}
\usepackage{algorithmic}
\usepackage{graphicx}
\usepackage{textcomp}
\usepackage{xcolor}
\usepackage{txfonts}
\usepackage{listings}
\usepackage{enumitem}
\usepackage{mathtools}
\usepackage{gensymb}
\usepackage{comment}
\usepackage[breaklinks=true]{hyperref}
\usepackage{tkz-euclide} 
\usepackage{listings}                                        
%\def\inputGnumericTable{}                                 
\usepackage[latin1]{inputenc}                                
\usepackage{color}                                            
\usepackage{array}                                            
\usepackage{longtable}                                       
\usepackage{calc}                                             
\usepackage{multirow}                                         
\usepackage{hhline}                                           
\usepackage{ifthen}                                           
\usepackage{lscape}
\usepackage{tabularx}
\usepackage{array}
\usepackage{float}
\usepackage{multicol}

\newcommand{\BEQA}{\begin{eqnarray}}
\newcommand{\EEQA}{\end{eqnarray}}
%\newcommand{\define}{\stackrel{\triangle}{=}}

\setlength{\headheight}{1cm} % Set the height of the header box
\setlength{\headsep}{0mm}     % Set the distance between the header box and the top of the text


%\usepackage[a5paper, top=10mm, bottom=10mm, left=10mm, right=10mm]{geometry}


\setlength{\intextsep}{10pt} % Space between text and floats

% Marks the beginning of the document
\begin{document}
\onecolumn
\bibliographystyle{IEEEtran}
\vspace{3cm}

%\renewcommand{\theequation}{\theenumi}
\numberwithin{equation}{enumi}
\numberwithin{figure}{enumi}
\renewcommand{\thefigure}{\theenumi}
\renewcommand{\thetable}{\theenumi}

\title{Mains.2.B.1-14}
\author{Shiven Bajpai}
\maketitle
\section{mains}
\fi

%\begin{enumerate}
	\item{$z$ and $w$ are two non zero complex numbers such that $\abs{z} = \abs{w}$ and $Arg\brak{z} + Arg\brak{w} = \pi$ then $z$ equals \hfill (2002)
		\begin{multicols}{4}
		\begin{enumerate}
			\item{$\overline{\omega}$} \columnbreak \item{$-\overline{\omega}$} \columnbreak \item{$\omega$} \columnbreak \item{$-\omega$}
		\end{enumerate}
		\end{multicols}
		}
		
	\item{If $\abs{z-4}<\abs{z-2}$, its solution is given by \hfill (2002)
		\begin{multicols}{4}
		\begin{enumerate}
			\item{$Re(z)>0$} 
			\columnbreak
			\item{$Re(z)<0$}
			\columnbreak
			\item{$Re(z)>3$}
			\columnbreak
			\item{$Re(z)>2$}
		\end{enumerate}
		\end{multicols}}
		
	\item{The locus of the centre of a circle which touches the circle $\abs{z-z_1}=a$ and $\abs{z-z_2}=b$ externally \brak{z, z_1, z_2 \text{ are complex numbers}} will be \hfill (2002)
		\begin{multicols}{4}
		\begin{enumerate}
			\item{an ellipse}
			\columnbreak
			\item{a hyperbola} 
			\columnbreak
			\item{a circle}
			\columnbreak
			\item{none of these}
		\end{enumerate}
		\end{multicols}}
		
	\item{If $z$ and $w$ are two non-zero complex numbers such that $\abs{zw}=1$ and $Arg\brak{z} - Arg\brak{w} = \frac{\pi}{2}$ then $\overline{z}w$ is equal to \hfill (2003)
		\begin{multicols}{4}
		\begin{enumerate}
			\item{$-\iota$}
			\columnbreak
			\item{$1$}
			\columnbreak
			\item{$-1$}
			\columnbreak
			\item{$\iota$}
		\end{enumerate}
		\end{multicols}}

	\item{Let $Z_1$ and $Z_2$ be two roots of the equation $Z^2 + aZ + b = 0$, $Z$ being complex. Further assume that the origin, $Z_1$ and $Z_2$ form an equilateral triangle. Then \hfill (2003)
		\begin{multicols}{4}
		\begin{enumerate}
			\item{$a^2 = 4b$}
			\columnbreak
			\item{$a^2 = b$}
			\columnbreak
			\item{$a^2 = 2b$}
			\columnbreak
			\item{$a^2 = 3b$}
		\end{enumerate}
		\end{multicols}}

	\item{If $\brak{\frac{1-\iota}{1+\iota}}^x = 1$ then \hfill (2003)
		\begin{enumerate}
			\item{$x = 2n + 1$, where n is any positive integer}
			\item{$x = 4n$, where n is any positive integer}
			\item{$x = 2n$, where n is any positive integer}
			\item{$x = 4n + 1$, where n is any positive integer}
		\end{enumerate}}
		
	\item{Let $z$ and $w$ be complex numbers such that $\overline{z} + \iota\overline{w} = 0$ and $arg\brak{zw} = \pi$ then $arg\brak{z}$ equals \hfill (2004)
		\begin{multicols}{4}
		\begin{enumerate}
			\item{$\frac{5\pi}{4}$}
			\columnbreak 
			\item{$\frac{\pi}{2}$}
			\columnbreak
			\item{$\frac{3\pi}{4}$}
			\columnbreak
			\item{$\frac{\pi}{4}$}
		\end{enumerate}
		\end{multicols}}

	\item{If $z=x-\iota y$ and $z^{\frac{1}{3}}=p+\iota q$, then \begin{align*} &\frac{\frac{x}{p} + \frac{y}{q}}{p^2 + q^2} \end{align*} is equal to 
		\hfill (2004)
		\begin{multicols}{4}
		\begin{enumerate}
			\item{$-2$}
			\columnbreak
			\item{$-1$}
			\columnbreak
			\item{$2$}
			\columnbreak
			\item{$1$}
		\end{enumerate}
		\end{multicols}}

	\item{If $\abs{z^2 - 1} = \abs{z}^2 + 1$, then $z$ lies on \hfill (2004)
		\begin{multicols}{2}
		\begin{enumerate}
			\item{an ellipse}
			\item{the imaginary axis}
			\columnbreak
			\item{a circle}
			\item{the real axis}
		\end{enumerate}
		\end{multicols}}
	

	\item{If the cube roots of unity are 1, $\omega$, $\omega^2$ then the roots of the equation $\brak{x-1}^3 + 8 = 0$, are \hfill (2004)
		\begin{multicols}{2}
		\begin{enumerate}
			\item{$-1,-1+2\omega,-1-2\omega ^2$}
			\item{$-1,-1,-1$}
			\columnbreak
			\item{$-1, 1-2\omega, 1-2\omega ^2$}
			\item{$-1, 1+2\omega, 1+2\omega ^2$}
		\end{enumerate}
		\end{multicols}}

	\item{If $z_1$ and $z_2$ are two non-complex numbers such that $\abs{z_1 + z_2} = \abs{z_1} + \abs{z_2}$, then arg\brak{z_1} - arg\brak{z_2} is equal to \hfill (2005)
		\begin{multicols}{4}
		\begin{enumerate}
			\item{$\frac{\pi}{2}$}
			\columnbreak
			\item{$-\pi$}
			\columnbreak
			\item{$0$}
			\columnbreak
			\item{$\frac{\pi}{2}$}
		\end{enumerate}
		\end{multicols}}

	\item{If \begin{align*} \omega &= \frac{z}{z-\frac{1}{3}\iota} \end{align*} and $\abs{\omega} = 1$, then $z$ lies on \hfill (2005)
		\begin{multicols}{4}
		\begin{enumerate}
			\item{an ellipse}
			\columnbreak
			\item{a circle}
			\columnbreak
			\item{a straight line}
			\columnbreak
			\item{a parabola}
		\end{enumerate}
		\end{multicols}}

	\item{The value of $\sum_{k=1}^{10}\brak{sin\brak{\frac{2k\pi}{11}}+\iota cos\brak{\frac{2k\pi}{11}}}$ is 
		\hfill (2006)
		\begin{multicols}{4}
		\begin{enumerate}
			\item{$\iota$}
			\columnbreak
			\item{$1$}
			\columnbreak
			\item{$-\iota$}
			\columnbreak
			\item{$-1$}
		\end{enumerate}
		\end{multicols}}

	\item{If $z^2 + z + 1 = 0$, where $z$ is a complex number, then the value of $$\brak{z+\frac{1}{z}}^2 + \brak{z^2+\frac{1}{z^2}}^2 +\brak{z^3+ \frac{1}{z^3}}^2 + \dots + \brak{z^6+\frac{1}{z^6}}^2 $$ is \hfill (2006)
		\begin{multicols}{4}
		\begin{enumerate}
			\item{$18$}
			\columnbreak
			\item{$54$}
			\columnbreak
			\item{$6$}
			\columnbreak
			\item{$12$}
		\end{enumerate}
		\end{multicols}}

%\end{enumerate}
%\end{document}
