\iffalse
\title{Assignment 3}
\author{AI24BTECH11031 - Shivram S}
\section{mains}
\fi

% \begin{enumerate}
    \setcounter{enumi}{30}
    \item $ABC$ is a triangle, right angled at $\vec A$. The resultant of
        the forces acting along $AB$. $BC$ with magnitudes $\frac {1} {AB}$
        and $\frac {1} {AC}$ respectively is the force along $AD$ where
        $\vec D$ is the foot of the perpendicular from $\vec A$ onto $BC$.
        The magnitude of the resultant is
        \hfill (2006)

        \begin{multicols}{2}
            \begin{enumerate}
                \item $\frac {AB^2 + AC^2} {(AB)^2(AC)^2}$
                \item $\frac {(AB)(AC)} {AB + AC}$
                \item $\frac {1} {AB} + \frac {1} {AC}$
                \item $ \frac {1} {AD}$
            \end{enumerate}
        \end{multicols}

    \item Let $W$ denote the words in the English dictionary. Define the
        relation $R$ by $R = \cbrak{\brak{x, y} \in W \times W \vert
        \text{ the words $x$ and $y$ have at least one letter in common.}}$
        then $R$ is
        \hfill (2006)

        \begin{enumerate}
            \item not reflexive, symmetric, transitive
            \item reflexive, symmetric and not transitive
            \item reflexive, symmetric and transitive
            \item reflexive, not symmetric and transitive
        \end{enumerate}

    \item Suppose a population $A$ has 100 observations 101, 102, $\dots$, 200
        and another population $B$ has 100 observations 151, 152, $\dots$, 250.
        If $V_A$ and $V_B$ represent the variances of the two populations,
        respectively then $\frac {V_A} {V_B}$ is
        \hfill (2006)

        \begin{multicols}{4} 
            \begin{enumerate}
                \item 1
                \item $\frac {9} {4}$
                \item $\frac {4} {9}$
                \item $\frac {2} {3}$
            \end{enumerate}
        \end{multicols}

    \item A particle has two velocities of equal magnitude inclined to
        each other at an angle $\theta$. If one of them is halved, the
        angle between the other and the original resultant velocity is
        bisected by the new resultant. Then $\theta$ is
        \hfill (2006)

        \begin{multicols}{4}
            \begin{enumerate}
                \item $90 \degree$
                \item $120 \degree$
                \item $45 \degree$
                \item $60 \degree$
            \end{enumerate}
        \end{multicols}

    \item A body falling from rest under gravity passes a certain
        point $\vec P$. It was at a distance of 400 m from $\vec P$,
        4s prior to passing through $\vec P$. If $g = 10 m/s^2$, then the
        height above the point $\vec P$ from where the body began to
        fall is
        \hfill (2006)

        \begin{multicols}{4}
            \begin{enumerate}
                \item 720 m
                \item 900 m 
                \item 320 m 
                \item 680 m 
            \end{enumerate} 
        \end{multicols}

    \item The resultant of two forces $Pn$ and $3n$ is a force of
        $7n$. If the direction of $3n$ force were reversed, the
        resultant would be $\sqrt{19}n$. The value of $P$ is
        \hfill (2007)

        \begin{multicols}{4}
            \begin{enumerate}
                \item $3n$
                \item $4n$ 
                \item $5n$
                \item $6n$ 
            \end{enumerate} 
        \end{multicols}

    \item A particle just clears a wall of height $b$ at a distance
        $a$ and strikes the ground at a distance $c$ from the point
        of projection. The angle of projection is
        \hfill (2007)

        \begin{multicols}{4}
            \begin{enumerate}
                \item $\tan^{-1} \frac {bc} {a(c - a)}$
                \item $\tan^{-1} \frac {bc} {a}$
                \item $\tan^{-1} \frac {b} {ac}$
                \item $45 \degree$
            \end{enumerate} 
        \end{multicols}

    \item The average marks of boys in class is 52 and that of girls
        is 42. The average marks of boys and girls combined is 50.
        The percentage of boys in the class is
        \hfill (2007)

        \begin{multicols}{4}
            \begin{enumerate}
                \item 80
                \item 60
                \item 40
                \item 20
            \end{enumerate} 
        \end{multicols}

    \item A body weighing 13 kg is suspended by two strings 5m and
        12 m long, their other ends fixed to the extremities of a
        rod 13 m long. If the rod be held so that the body hangs
        immediately below the middle point then the tensions in the
        strings are:
        \hfill (2007)

        \begin{multicols}{2}
            \begin{enumerate}
                \item 5 kg and 12 kg
                \item 5 kg and 13 kg
                \item 12 kg and 13 kg
                \item 5 kg and 5 kg
            \end{enumerate} 
        \end{multicols}

    \item The mean of the numbers $a$, $b$, 8, 5, 10 is 6 and
        the variance is $6.80$. Then which one of the following
        gives the possible values of $a$ and $b$?
        \hfill (2008)

        \begin{multicols}{2}
            \begin{enumerate}
                \item $a = 0$, $b = 7$
                \item $a = 5$, $b = 2$
                \item $a = 1$, $b = 6$
                \item $a = 3$, $b = 4$
            \end{enumerate}
        \end{multicols}

    \item Let $p$ be the statement ``$x$ is an irrational number'',
        $q$ be the statement ``$y$ is a transcendental number'',
        and $r$ be the statement ``$x$ is a rational number if
        $y$ is a transcendental number''.
        \hfill (2008)

        \textbf{Statement-1}: $r$ is equivalent to either $q$ or $p$ \\
        \textbf{Statement-2}: $r$ is equivalent to $\sim \brak{p \leftrightarrow \sim q}$

        \begin{enumerate}
            \item Statement-1 is false, Statement-2 is true
            \item Statement-1 is true, Statement-2 is true;
                Statement-2 is a correct explanation for Statement-1
            \item Statement-1 is true, Statement-2 is true;
                Statement-2 is not a correct explanation for Statement-1
            \item Statement-1 is true, Statement-2 is false
        \end{enumerate}

    \item The statement $p \to \brak{q \to p}$ is equivalent to
        \hfill (2008)

        \begin{multicols}{2}
            \begin{enumerate}
                \item $p \to \brak{p \to q}$
                \item $p \to \brak{p \lor q}$
                \item $p \to \brak{p \land q}$
                \item $p \to \brak{p \leftrightarrow q}$
            \end{enumerate}
        \end{multicols}
        
    \item \textbf{Statement-1}: $\sim \brak{p \leftrightarrow \sim q}$ is
        equivalent to $p \leftrightarrow q$ \\
        \textbf{Statement-2}: $\sim \brak{p \leftrightarrow \sim q}$
        is a tautology
        \hfill (2009)
        
        \begin{enumerate}
            \item Statement-1 is true, Statement-2 is true;
                Statement-2 is not a correct explanation for Statement-1
            \item Statement-1 is true, Statement-2 is false
            \item Statement-1 is false, Statement-2 is true
            \item Statement-1 is true, Statement-2 is true;
                Statement-2 is a correct explanation for Statement-1       
        \end{enumerate}

    \item \textbf{Statement-1}: The variance of the first $n$ even
        natural numbers is $\frac {n^2 - 1} {4}$. \\
        \textbf{Statement-2}: The sum of first $n$ natural numbers
        is $\frac {n(n + 1)} {2}$ and the sum of squares of first
        $n$ natural numbers is $\frac {n(n+1)(2n+1)} {6}$.
        \hfill (2009)

        \begin{enumerate}
            \item Statement-1 is true, Statement-2 is true;
                Statement-2 is not a correct explanation for Statement-1
            \item Statement-1 is true, Statement-2 is false
            \item Statement-1 is false, Statement-2 is true
            \item Statement-1 is true, Statement-2 is true;
                Statement-2 is a correct explanation for Statement-1       
        \end{enumerate}

    \item If $A$, $B$ and $C$ are three sets such that $A \cap B = A \cap C$
        and $A \cup B = A \cup C$, then
        \hfill (2009)
        
        \begin{multicols}{2}
            \begin{enumerate}
                \item $A = C$
                \item $B = C$
                \item $A \cup B = \emptyset$
                \item $A = B$
            \end{enumerate}
        \end{multicols}
% \end{enumerate}
