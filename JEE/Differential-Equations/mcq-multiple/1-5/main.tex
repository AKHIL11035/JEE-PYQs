\iffalse
\title{CHAPTER - 19\\Differential Equations}
\author{AI24BTECH11028 - Ronit Ranjan}
\section{mcq-multiple}
\fi

%\begin{enumerate}
    
\item The order of the differential equation whose general solution is given by
$y = \brak{C_1 + C_2} \cos \brak{x + C_3} - C_4 e^{x+C_5}$, where $C_1, C_2, C_3, C_4, C_5$ are arbitrary constants, is \hfill \brak{1998 - 2 Marks}

\begin{multicols}{2}
\begin{enumerate}
    \item $5$
    \item $3$ 
    \item $4$
    \item $2$ 
\end{enumerate} 
\end{multicols}

\item  The differential equation representing the family of curves
$y^2 = 2c \brak{x + \sqrt{c}}$, where c  is a positive parameter, is of \hfill \brak{1999 - 3 Marks}

\begin{multicols}{2}
\begin{enumerate}
    \item order $1$ 
    \item degree $3$
    \item order $2$
    \item degree $4$
\end{enumerate}
\end{multicols}

\item  A curve $ y = f(x) $ passes through $ \brak{1,1} $ and at $ P\brak{x, y} $, the tangent cuts the $ x $-axis and $ y $-axis at $ A $ and $ B $ respectively such that $ BP : AP = 3 : 1 $, then \hfill \brak{2006- 5M,-1}
\begin{enumerate}
    \item equation of curve is $ xy^\prime - 3y = 0 $
    \item normal at $ \brak{1, 1} $ is $ x + 3y = 4 $
    \item curve passes through $ \brak{2, \frac{1}{8}} $
    \item equation of curve is $xy^\prime + 3y = 0 $
\end{enumerate}

\item  If $y\brak{x}$ satisfies the differential equation $y^\prime - y \tan x = 2x \sec x $ and $y\brak{0} = 0 $, then \hfill \brak{2012}

\begin{multicols}{2}
\begin{enumerate}
    \item $ y\brak{\frac{\pi}{4}} = \frac{\pi^2}{8\sqrt{2}} $
    \item $ y\brak{\frac{\pi}{3}} = \frac{\pi^2}{9} $
    \item $ y\brak{\frac{\pi}{4}} = \frac{\pi^2}{18} $
    \item $ y\brak{\frac{\pi}{3}} = \frac{4\pi}{3} + \frac{2\pi^2}{3\sqrt{3}} $
\end{enumerate}
\end{multicols}

\item A curve passes through the point $ \brak{1,\frac{\pi}{6}} $. Let the slope of the curve at each point $ \brak{x, y} $ be $ \frac{y}{x} + \sec\brak{\frac{y}{x}}, x > 0 $. Then the equation of the curve is \hfill \brak{JEE Adv. 2013}

\begin{multicols}{2}
\begin{enumerate}
    \item $ \sin\brak{\frac{y}{x}} = \log x + \frac{1}{2} $
    \item $\sec\brak{\frac{2y}{x}} = \log x + 2$
    \item $\cos\sec\brak{\frac{y}{x}} = \log x + 2 $
    \item $ \cos\brak{\frac{2y}{x}} = \log x + \frac{1}{2} $
\end{enumerate}
\end{multicols}
%\end{enumerate}
