\iffalse
  \title{conic-sections}
  \author{KOTHAPALLI AKHIL}
  \section{integer}
\fi
%\begin{enumerate}
\item The line $2x+y=1$ is the tangent to the hyperbola $\frac{x^2}{a^2}-\frac{y^2}{b^2}=1$. If this line passes through the point of intersection of the nearest directrix and the X-axis, then the eccentricity of the hyperbola is
\hfill(2010)
\item Consider the parabola $y^2=8x$ . Let $\Delta_1$ be the area of the triangle formed by the end points of its latus rectum and the point $\Vec{P}$$(\frac{1}{2},2)$ on the parabola and $\Delta_2$ be the area of the triangle formed by drawing tangents at $\Vec{P}$ and at the end points of the latus rectum.Then $\frac{\Delta_1}{\Delta_2}$ is 
\hfill(2011)
\item Let $\Vec{S}$ be the focus of the parabola $y^2=8x$ and let $PQ$ be the common chord of the circle $x^2+y^2-2x-4y=0$ and the given parabola. The area of the triangle $PQS$ is
\hfill(2012)
\item A Vertical line passing through point $(h,0)$ intersects the ellipse   at the points  $\Vec{P}$ and $\Vec{Q}$ . Let the tangents to the ellipse at $\Vec{P}$ and $\Vec{Q}$ meet at the points $\Vec{R}$.If $\Delta(h)$= area of the triangle $PQR$, $\Delta_1$= ma
then 
\hfill(JEE Adv.2013)
\begin{enumerate}
    \item $g(x)$ is continuous but not differentiable at a
    \item $g(x)$ is differentiable on R
    \item $g(x)$ is continuous but not differentiable at b
    \item $g(x)$ is continuous and differentiable either(a) or (b) but not both 
    \end{enumerate}
\item If the normal of the parabola $y^2=4x$ drawn at the end points of its latusrectum are the tangents of th circle $(x-3)^2+(y+2)^2=r^2$, then the value of $r^2$ is
\hfill(JEE Adv.2015)
\item Let the curve C be the mirror image of the parabola $y^2=4x$ with respect to the line $x+y+4=0$.If $\Vec{A}$ and $\Vec{B}$ are the points of the intersection of C with the line $y=-5$,then the distance between $\Vec{A}$and $\Vec{B}$ is
\hfill(JEE Adv.2015)
\item Suppose that the focii of the ellipse $\frac{x^2}{9}+\frac{y^2}{5}=1$ are $(f_1,0)$ and ($f_2$,0) where $f_1>0$ and $f_1<0$.Let $P_1$ and $P_2$ be two parabolas with a common vertex at $(0,0)$ and with foci at ($f_1$,0) and (2$f_2$,0),respectively. Let $T_1$ be a tangent to $P_1$ which passes through (2$f_2$,0) and $T_2$ be a tangent to $P_2$ which passes through ($f_1$,0).If $m_1$ is the slope of $T_1$ and $m_2$ is the slope of $T_2$,then the value of
\hfill(JEE Adv. 2015)
%\end{enumerate}

% \end{document}
