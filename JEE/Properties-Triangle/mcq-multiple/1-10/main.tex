\iffalse
\title{Properties of Triangles}
\author{EE24Btech11022 - Eshan sharma}
\section{mcq-multiple}
\fi
%begin{document}
%begin{enumerate}
    \item There exists a triangle $\vec{ABC}$ satisfying the conditions
    \hfill{(1986 - 2 mark)}
    \begin{enumerate}
    \item $b\sin{A}$ $=$ a, A $<\pi/2$
    \item $b\sin{A}$ $>$ a, A $>\pi/2$
    \item $b\sin{A}$ $>$ a, A $<\pi/2$
    \item $b\sin{A}$ $<$ a, A $<\pi/2$, b $>$ a
    \item $b\sin{A}$ $<$ a, A $>\pi/2$, b $=$ a
    \end{enumerate}
    \item In a triangle, the lengths of two larger sides are 10 and 9 respectively. If the angles are in AP, Then length of third side is
    \hfill{(1987 - 2 mark)}
    \begin{multicols}{2}
    \begin{enumerate}
    \item $5-\sqrt{6}$ 
    \item $3\sqrt{3}$
    \item $3$
    \item $5+\sqrt{6}$ 
    \item none
    \end{enumerate}
    \end{multicols}
    \item If in a triangle PQR, $sinP, sinQ, sinR$ are in AP, then
    \hfill{(1998 - 2 mark)}
    \begin{enumerate}
    \item The altitudes are in AP
    \item The altitudes are in HP
    \item The medians are in GP
    \item The medians are in AP
    \end{enumerate}
    \item Let $A_{0}A_{1}A_{2}A_{3}A_{4}A_{5}$ be a regular hexagon inscribed in a circle of unit radius. Then the product of the lengths of the line segments $A_{0}A_{1}$,$A_{0}A_{2}$ and $A_{0}A_{4}$ is 
    \hfill{(1998 - 2 mark)}
    \begin{multicols}{2}
    \begin{enumerate}
    \item ${\frac{3}{4}}$
    \item $3\sqrt{3}$
    \item 3
    \item ${\frac{3\sqrt{3}}{2}}$
    \end{enumerate}
    \end{multicols}
    \item In $\Delta$ABC, internal angle bisector of $\angle A$ meets side BC in $\vec{D}$. DE $\perp$ AD meets AC in $\vec{E}$ and AB in $\vec{F}$. Then
    \hfill{(2006-5M,-1)}
    \begin{enumerate}
    \item AE is HM of b and c
    \item AD = ${\frac{2bc}{b+c}}cos{\frac{A}{2}}$
    \item EF = ${\frac{4bc}{b+c}}sin{\frac{A}{2}}$
    \item $\Delta$AEF is isosceles
    \end{enumerate}
    \item Let ABC be a triangle such that $\angle ACB = \pi/6$ and let a,b and c denote lengths of the sides opposite to $\vec{A}$,$\vec{B}$ and $\vec{C}$ respectively. The value(s) of $x$ for which $a = x^{2}+x+1, b = x^{2}-1, c = 2x+1$ is(are)
    \hfill{(2010)}
    \begin{multicols}{2}
    \begin{enumerate}
    \item $-(2+\sqrt{3})$
    \item $1+\sqrt{3}$
    \item $2+\sqrt{3}$
    \item $4\sqrt{3}$
    \end{enumerate}
    \end{multicols}
    \item In a triangle PQR, $\vec{P}$ is the largest angle and $cosP = \frac{1}{3}$. Further the incircle of the triangle touches the sides PQ,QR and RP at $\vec{N},\vec{L} and \vec{M}$ respectively, such that the lengths of PN, QL and RM are consecutive even integers. Then possible length(s) of the side(s) of the triangle is(are)
    \hfill{(Jee Adv. 2013)}
    \begin{multicols}{2}
    \begin{enumerate}
    \item 16
    \item 24
    \item 18
    \item 22
    \end{enumerate}
    \end{multicols}
    \item In a triangle XYZ, let $x,y,z$ be the lengths of sides opposite to angles $\vec{X},\vec{Y},\vec{Z}$ and $2s = x+y+z$. If ${\frac{s-x}{4}}={\frac{s-y}{3}}={\frac{s-z}{2}}$ and area of the incircle of the triangle XYZ is ${\frac{8\pi}{3}}$
    \hfill{(Jee Adv. 2016)}
    \begin{enumerate}
    \item area of the triangle is 6$\sqrt{6}$
    \item the radius of circumcirle of XYZ is ${\frac{35\sqrt{6}}{6}}$
    \item $sin\frac{X}{2}sin\frac{Y}{2}sin\frac{Z}{2} = \frac{4}{35}$
    \item $sin^{2}\brak{\frac{X+Y}{2}}$ = $\frac{3}{5}$
    \end{enumerate}
    \item In a triangle PQR, let $\angle PQR = 30\degree$ and the sides PQ and QR have lengths $10\sqrt{3}$ and 10 respectively. Then which of the following statements is(are) TRUE?
    \hfill{(Jee Adv. 2018)}
    \begin{enumerate}
    \item $\angle QPR = 45\degree$
    \item the area of the triangle PQR is $25\sqrt{3}$ and $\angle QRP = 120\degree$
    \item the radius of the incircle of triangle PQR is $10\sqrt{3}-15$
    \item the radius of circumcirle PQR is $100\pi$
    \end{enumerate}
    \item In a non-right-angle triangle $\Delta PQR$, let p,q,r denote the lengths of the sides opposite to the angles at $\vec{P},\vec{Q},\vec{R}$ respectively. The median from $\vec{R}$ meets the side PQ at $\vec{S}$, the perpendicular from $\vec{P}$ meets the side QR at $\vec{E}$, RS and PE
	    intersect at $\vec{O}$. If p = $\sqrt{3}$, q = 1 and the radius of the circumcircle at $\Delta PQR$ equals 1, then which of the following options is(are)\\ correct.
    \hfill{(Jee Adv. 2018)}
    \begin{enumerate}
    \item Radius of incircle of $\Delta PQR$ = $\frac{\sqrt{3}}{2}\brak{2-\sqrt{3}}$
    \item Area of $\Delta SOE = \frac{\sqrt{3}}{12}$
    \item Length of OE = $\frac{1}{6}$
    \item Length of RS = $\frac{\sqrt{7}}{2}$
    \end{enumerate}

%end{enumerate}
%end{document}

